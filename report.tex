\documentclass[runningheads]{llncs}
%Packages---------------------------------
\usepackage{bm}
\usepackage{amsmath}
%Macro------------------------------------
\def\MPC{\mathsf{MPC}}
\def\NIMPC{\mathsf{NIMPC}}
\def\BITMPC{\mathsf{BIT-MPC}}
\def\bitset{\{0,1\}}
\def\view{\mathsf{View}}
\def\eval{\mathsf{Eval}}
\def\msg{\mathsf{Msg}}
\def\gen{\mathsf{Gen}}
\def\compile{\mathsf{Compile}}
\def\psigen{\Psi_{\gen}}
\def\psieval{\Psi_{\eval}}
%Title-------------------------------------
\title{Difficulty of $\BITMPC$ for 5-input function from the Compiler}
\author{{Protik Kumar Paul}}
\institute{IISc, CSA}
%\Project{Secure Computation Project}
\date{\today}
\begin{document}

\maketitle

\begin{abstract}
Best Possible Information-Theoretic Security for Multi-Party Computation ($\BITMPC$) was presented in \cite{BITMPC} which addressed the problem that what is the best possible notion of security that one can hope to achieve in the Information-Theoretic(IT) setting. A new notion of secuity namely Best-possible security is defined. In the paper a complier was presented for a function $f(\cdot)$ for which there is a Non-Interactive Multi-Party Computation ($\NIMPC$) \cite{NIMPC} protocol and there are secure MPC to generate the correlated randomness and secure MPC to emulate the evaluator of the $\NIMPC$. 
\end{abstract}

\section{Introduction}
From the impossibility result in \cite{Cramer98} we know that in dishonest majority setting perfect secrecy cannot be achieved. So any protocol which provides perfect secrecy in honest majority does not provide any security guarantee when the number of courrupt parties is more than or equal to total number of parties in the protocol. In the paper \cite{BITMPC} notion of security defined as follows: A protocol $\Pi$ for an $n$ input function $f(\cdot)$ is called $t-\BITMPC$ if $\Pi$ provides standard security if the number of corrupt parties is $\leq t$ and provides residual security if the number of corrupt parties is $> t$. Standard security says that the adversary is allowed to learn the output and any information which can be computed from the adversary's input and output, whereas Residual security means that adversary is allowed to learn the residual function of $f$ on the corrupt set and in the input.

\textbf{Residual Function:} Consider a fixed $n$-input function $f:(\bitset^*)^n\rightarrow\bitset^*$, let $x=(x_1,\ldots,x_n)$ be an input to $f$, and let $T=\{i_1,\ldots,l_{n'}\}\subseteq [n]$ be a subset of size $n'$. The residual function for $T$ and $x$ is an $n'$-input function $f_{T,x}:(\bitset^*)^{n'}\rightarrow \bitset^*$, obtained from $f$ by restricting the input variables indexed by $[n]\setminus T$ to their values in $x$. i.e. $f_{T,x}(y_1,\ldots,y_{n'})=f(z_1,\ldots,z_n)$, where for $l\notin T$ we have $z_l=x_l$, while for $l=i_j \in T$ we have $z_l=y_l$.

\textbf{Residual Security:} Let $f$ be an $n$-input function, let $\Pi$ be an $n$-party protocol, for parties $P_1,\ldots P_n$ and $T\subseteq [n]$ be the set of corrupt parties. We say $\Pi$ provides residual security against $T$ if for any two inputs $x, x'$ such that $x_T=x'_T$ and the residual function $f_{T,x} \equiv f_{T,x'}$, the two views $\view_T(x)=\{\{\view_{P_i}(x)\}_{i\in T}, f_{T,x}\}$ and $\view_T(x')=\{\{\view_{P_i}(x')\}_{i\in T}, f_{T,x'}\}$ are statistically close.

\textbf{Standard Security:} Let $f$ be an $n$-input function, let $\Pi$ be an $n$-party protocol, for parties $P_1,\ldots P_n$ and $T\subseteq [n]$ be the set of corrupt parties. We say $\Pi$ provides standard security against $T$ if for any two inputs $x, x'$ such that $x_T=x'_T$ and $f(x)=f(x')$, the two views $\view_T(x)=\{\{\view_{P_i}(x)\}_{i\in T}, f(x)\}$ and $\view_T(x')=\{\{\view_{P_i}(x')\}_{i\in T}, f(x')\}$ are statistically close.

$\bm{\BITMPC$} Let $f$ be an $n$-input function, let $\Pi$ be an $n$-party protocol and consider some threshold $t\leq n$. We say that $\Pi$ is a $t$-private, best-possible, information-theoretic protocol for $f$ $(t-\BITMPC)$ if the following conditions hold:
	\begin{itemize}
	\item Correctness: For all $x\in (\bitset^*)^n$ it holds that $					\Pi(x)=f(x)$ with all but negligible probability.
	\item For any set $T\subseteq [n]$, $|T|\leq t$, $\Pi$ provides 				standard security against $T$.
	\item For any set $T\subseteq [n]$, $|T|> T$, $\Pi$ provides residual 			security against $T$.
\end{itemize}

Now we will present the idea of \cite{BITMPC} to obtain a $\BITMPC$ protocol form $\NIMPC$. To do that let us recall the notion of $\NIMPC$.

In the $\NIMPC$ parties are not allowed to interact. To provide the security in this setting, the parties are provided with some \textit{correlated randomness}, which is chosen ahead of time, independently of the secret inputs. With this setup, each party sends a single message, which is generated from his secret input and the correlated randomness, to a designated party (other than $P_1,\ldots,P_n$) $P_0$, called evaluator. Evaluator on all parties messages runs $\eval$ algorithm to obtain the output $f(x_1,\ldots,x_n)$. Note that some of the parties may collude with the evaluator, in this adversial model one cannot hope to get anything better than the residual security, as adversary may make the colluding parties to send all the messages corresponding to all the secret input, therefore the evaluator can run $\eval$ on multiple inputs where inputs of the honest parties may vary, which is more than the standard secuirty. But leaking this information is allowed in the residual security.



\section{Compiler}
In the paper \cite{BITMPC}, a compiler is presented which starts with an $\NIMPC$ protocol for a function $f$, and constructs a $\BITMPC$ protocol for $f$. In more detail, the ingredients for the compiler are protocols for parties $P_1,\ldots, P_n$: 
\begin{itemize}
\item An $n$-party $\NIMPC$ protocol $\Pi=\{\gen,\msg,\eval\}$ for an $n$-input function $f$.
\item An $n$-party interactive $\MPC$ protocol $\psigen$ for the randomized $\gen$ function of $\Pi$.
\item An $n$-party interactive $\MPC$ protocol $\psieval$ for the $\eval$ function of $\Pi$.
\end{itemize}
Given these protocols, the resulting interactive protocol $\Phi = \compile(\Pi, \psigen,\psieval)$ is as follows. On inputs $x_1,\ldots, x_n$ held by $P_1,\ldots, P_n$ respectively:
\begin{enumerate}
\item The parties run $\psigen$ to evaluate $\gen$; $r_i$ is the output of party $P_i$;
\item Each party $P_i$ computes locally $m_i=\msg(x_i,r_i)$;
\item The parties run $\psieval$ each using $m_i$ as its input to the protocol, to get $y=\eval(m_1,\ldots,m_n)$.
\end{enumerate}
\begin{theorem}
	Let $f$ be an $n$-input function, $\Pi$ a private $\NIMPC$ protocol for $f$, and let $\psigen$, $\psieval$ and the resulting $\Phi=\compile(\Pi, \psigen, \psieval)$ be as above.
	Correctness: If $\psigen$  and $\psieval$ are correct then $\Phi$ is a correct protocol for $f$.
	Security: For any subset $T\subseteq [n]$, the following holds:
	\begin{itemize}
	\item Residual Security: If $\psigen$ is correct and provides standard security against $T$, then $\Phi$ provides (at least) residual security against $T$.
	\item Standard security: If $\psigen$ is correct and $\psieval$ is correct and provides standard security against $T$ then the resulting $\Phi$ also provides standard security against $T$. 
	\end{itemize}
\end{theorem}

Using the above compiler, in the paper \cite{BITMPC}, another positive result is given that says that: 
\begin{theorem}
For every 4-input function $f$, there is a $1-\BITMPC$ interactive protocol for computing $f$.
\end{theorem}
\begin{proof}
Let us assume that $P_1,P_2,P_3,P_4$ are parties who has input $x_1,x_2,x_3,x_4$ respectively and they want to compute $f(x_1,x_2,x_3,x_4)$.\\
Let $\Pi=(\gen, \msg, \eval)$ be an $\NIMPC$ protocol for $f$ with general correlated randomness and interactive protocol for $\gen, \eval$ is as follows:
	\begin{itemize}
	\item For $\psigen$, we use a 1-of-3 BGW protocol, run by $P_2, P_3, P_4$, to generate the needed correlated randomness.
	\item For $\psieval$, we use a 2-of-5 BGW protocol for evaluation, where $P_2,P_3,P_4$ each play a single party, and $P_1$ plays the role of two parties.
	\end{itemize}
	\textbf{If the number of corrupt parties is 1:} then both $\psigen$ and $\psieval$ provide standard security and hence by the construction of the compiler, $\Phi$ provides standard security against any set of corrupt parties with number of corrupt parties is 1.\\
	\textbf{If the number of corrupt parties is 2:} If $P_1$ is corrupt then $\psieval$ is not secure but in $\psigen$ one party is corrupt, since 1 corruption is allowed in $\psigen$ it is secure, hence $\Phi$ is residual secure. If $P_1$ is not corrupt then two of $P_2,P_3,P_4$ are corrupt, which makes $\psigen$ insecure but $\psieval$ provides standard security against 2 corrupt parties, therefore $\psieval$ provides standard security, hence $\Phi$ has standard security.\\
	\textbf{If the number of corrupt parties is 3:} Suppose $P_1,P_2,P_3$ are corrupt. Then they are allowed to learn the residual function. The residual function does not reveal any information about $P_4$'s input only if the function is independent of $P_4$'s input, this assumption implies a strict condition on the function. Therefore for any general fucntion the residual function of a 4-input function restricted on a set of size 3, reveals the input of the $4^{th}$ party. So security arguement is not required in this case. Note that this can be generalized to the fact that if $n-1$ parties are corrupt then residual security allows the adversary to learn the honest party's input.
\end{proof}


\section{Question}
\subsection{Why 4?}
What is sacred about 4 such that solution worked? Can a similar construction done for any other $n$? Our first approach is to find a similar combination for n=5 such that either $\psigen$ or $\psieval$ is secure for corruption upto 3.
\subsection{Difficulties for 5}
In the beginning to satisfy the above constraints good number of random combinations were tried, but none of them satisfy the constraints.\\ %for example consider the following:\\
Then next approach is to abstract out the problem in a system of linear inequality.\\
Note that in the solution given in \cite{BITMPC} for 4-input function, for $\psigen$ parties were $P_2,P_3,P_4$, that can be viewed as $P_1$ with weight 0, $P_2,P_3,P_4$ with weight 1. Similarly, for $\psieval$ parties were $P_1, P_1, P_2,P_3,P_4$, which can be viewed as $P_1$ with weight 2 and $P_2,P_3,P_4$ with weight 1. Let $WtVecf_i$ is the weight of party $P_i$ for a protocol for the function $f(\gen/\eval$).\\
Now consider $p_i$ as a boolean variable such that the party $P_i$ is honest iff $p_i=1$. Therefore we can say a protocol ($\psigen/\psieval$) for a function ($\gen /\eval$) is secure if 
$$ \sum_{i\in[n]} WtVecf_i \times p_i > (\sum_{i\in[n]} WtVecf_i)/2$$.
Therefore the whole problem can be viewed as:
$$ \sum_{i\in[n]} WtVec\gen_i \times p_i > (\sum_{i\in[n]} WtVec\gen_i)/2$$ or
$$ \sum_{i\in[n]} WtVec\eval_i \times p_i > (\sum_{i\in[n]} WtVec\eval_i)/2$$
for all $(p_1,\ldots, p_n)$ such that number of zeros is $\leq n-2$.
Therefore in the above system of linear inequality there are $2n$ variables( $n$ for $WtVec\gen$ and $n$ for $WtVec\eval$) and $(\binom{n}{1} + \binom{n}{2} +  \ldots + \binom{n}{n-2})$ many constraints as this is the total number of all possible binary strings of length $n$ with number of zeros $\leq n-2$.
As the number of variables is much smaller than the number of constraints, so it's very unlikely that the above system of linear inequality will have any solution.\\
If there is not any solution for this system of linear inequality then we can say that constructing a similar compiler for arbitrary n for any function is not possible.\\
To check this for $n=5$ we have implemented and the result is that there is no combination till weight for all parties are 4.

\bibliographystyle{plain}
\bibliography{references}
\end{document}\grid
\grid
\grid
\grid
