\section{Introduction}
From the impossibility result in \cite{Cramer98} we know that in dishonest majority setting perfect secrecy cannot be achieved. So any protocol which provides perfect secrecy in honest majority does not provide any security guarantee when the number of courrupt parties is more than or equal to total number of parties in the protocol. In the paper \cite{BITMPC} notion of security defined as follows: A protocol $\Pi$ for an $n$ input function $f(\cdot)$ is called $t-\BITMPC$ if $\Pi$ provides standard security if the number of corrupt parties is $\leq t$ and provides residual security if the number of corrupt parties is $> t$. Standard security says that the adversary is allowed to learn the output and any information which can be computed from the adversary's input and output, whereas Residual security means that adversary is allowed to learn the residual function of $f$ on the corrupt set and in the input.

\textbf{Residual Function:} Consider a fixed $n$-input function $f:(\bitset^*)^n\rightarrow\bitset^*$, let $x=(x_1,\ldots,x_n)$ be an input to $f$, and let $T=\{i_1,\ldots,l_{n'}\}\subseteq [n]$ be a subset of size $n'$. The residual function for $T$ and $x$ is an $n'$-input function $f_{T,x}:(\bitset^*)^{n'}\rightarrow \bitset^*$, obtained from $f$ by restricting the input variables indexed by $[n]\setminus T$ to their values in $x$. i.e. $f_{T,x}(y_1,\ldots,y_{n'})=f(z_1,\ldots,z_n)$, where for $l\notin T$ we have $z_l=x_l$, while for $l=i_j \in T$ we have $z_l=y_l$.

\textbf{Residual Security:} Let $f$ be an $n$-input function, let $\Pi$ be an $n$-party protocol, for parties $P_1,\ldots P_n$ and $T\subseteq [n]$ be the set of corrupt parties. We say $\Pi$ provides residual security against $T$ if for any two inputs $x, x'$ such that $x_T=x'_T$ and the residual function $f_{T,x} \equiv f_{T,x'}$, the two views $\view_T(x)=\{\{\view_{P_i}(x)\}_{i\in T}, f_{T,x}\}$ and $\view_T(x')=\{\{\view_{P_i}(x')\}_{i\in T}, f_{T,x'}\}$ are statistically close.

\textbf{Standard Security:} Let $f$ be an $n$-input function, let $\Pi$ be an $n$-party protocol, for parties $P_1,\ldots P_n$ and $T\subseteq [n]$ be the set of corrupt parties. We say $\Pi$ provides standard security against $T$ if for any two inputs $x, x'$ such that $x_T=x'_T$ and $f(x)=f(x')$, the two views $\view_T(x)=\{\{\view_{P_i}(x)\}_{i\in T}, f(x)\}$ and $\view_T(x')=\{\{\view_{P_i}(x')\}_{i\in T}, f(x')\}$ are statistically close.

$\bm{\BITMPC$} Let $f$ be an $n$-input function, let $\Pi$ be an $n$-party protocol and consider some threshold $t\leq n$. We say that $\Pi$ is a $t$-private, best-possible, information-theoretic protocol for $f$ $(t-\BITMPC)$ if the following conditions hold:
	\begin{itemize}
	\item Correctness: For all $x\in (\bitset^*)^n$ it holds that $					\Pi(x)=f(x)$ with all but negligible probability.
	\item For any set $T\subseteq [n]$, $|T|\leq t$, $\Pi$ provides 				standard security against $T$.
	\item For any set $T\subseteq [n]$, $|T|> T$, $\Pi$ provides residual 			security against $T$.
\end{itemize}

Now we will present the idea of \cite{BITMPC} to obtain a $\BITMPC$ protocol form $\NIMPC$. To do that let us recall the notion of $\NIMPC$.
In $\NIMPC$ for an $n$-input function $f$, let $P_1,\ldots, P_n$ parties are there where $P_i$ has input $x_i$ and another party $P_0$ other than the above parties who is the evaluator for the protocol $\Pi$, then $\Pi$ is an $\NIMPC$ if 
In the $\NIMPC$ parties are not allowed to interact. To provide the security in this setting, the parties are provided with some \textit{correlated randomness}, which is chosen ahead of time, independently of the secret inputs. With this setup, each party sends a single message, which is generated from his secret input and the correlated randomness, to a designated party (other than $P_1,\ldots,P_n$) $P_0$, called evaluator. Evaluator on all parties messages runs $\eval$ algorithm to obtain the output $f(x_1,\ldots,x_n)$